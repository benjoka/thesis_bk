\section{Introduction}
\label{cha:introduction}
When the SARS-CoV-2 pandemic emerged in 2019, the world was faced with the immense challenge of understanding and containing a new virus. Local health authorities were overwhelmed by the effort required to identify contacts and outbreak scenarios, a process that relied mainly on interviewing infected individuals.
This global pressure, coupled with the recent expansion of genomic sequencing technologies, has resulted in increased genome sequencing efforts around the world. Observing the genomic structure, and particularly its evolutionary characteristics, provides biological evidence and insight into infection chains. This enables a rapid public health response to changing viral outbreak dynamics, as was demonstrated during the SARS-CoV-2 pandemic. However, genomic data analysis is a complex field that is not yet very user-oriented due to the immense runtime of its underlying algorithms and the high level of expertise required. In scenarios with greater availability of genome sequences, such as in large geographic areas or over longer seasonal periods, the time required for these processes exceeds the timeframe within which public health responses must be decided.

The GENTRAIN project is just one example of a user-oriented application that emerged from the exceptional situation caused by the SARS-CoV-2 pandemic \cite{Fra1}. The project was launched by the state government of North Rhine-Westphalia and funded by the European Union to support local public health authorities in conducting outbreak analyzes that combine contact tracing data with visualizations of the underlying infection chains. These visualizations are based on genetic distance matrices, which are made up of genetic distances between all isolates in a given dataset. While the calculation of these matrices is relatively efficient for confined municipal scenarios, it is not well-suited for large-scale scenarios due to its quadratic scaling with the number of genomes observed. Furthermore, GENTRAIN attaches great importance to the accurate determination of low genetic distances, particularly because its focus is on the identification and tracking of infection chains. This requires a more complex algorithm to calculate a single genetic distance, which results in longer computation times for individual calculations.

The objective of this work is to approach this problem scientifically by providing solutions that allow for the analysis of viral outbreaks in larger but still plausible scenarios.
Given the direct proportionality between the calculation time of genetic distance matrices and the number of sequences considered, the methodological approach of this work is designed to identify and consider only the most relevant distances in a genetic distance matrix. This approach was taken with a focus on preserving the interpretability of the visualization to the greatest extent feasible.
The subsequent research question is therefore posed: 

\vspace{10pt}
\noindent
\textit{"How can the computational effort involved in visualizing large-scale viral outbreak dynamics be minimized while preserving interpretability for outbreak analysis?"}
\vspace{10pt}

The preservation of interpretability for outbreak analysis was evaluated with respect to the representation of infection chains, mutation dynamics, and seasonal and geographic contexts. This work neither examines acceptance criteria for user-oriented applications in a bioinformatics environment nor claims to provide the most efficient method for calculating large-scale genetic distance matrices. The objective of this work is instead to evaluate the potential of these methods in the context of outbreak analysis.
This work focused on the algorithm used by GENTRAIN to calculate genetic distance. Great care was taken to ensure that the results could be transferred to other implementations. The same applies to this work's focus on the SARS-CoV-2 virus, which was selected because of the abundance of available sequences caused by the pandemic.

To minimize the required runtime, several approaches were carried out. First, the calculation of a single genetic distance implemented by GENTRAIN was modified to decrease its runtime. Then, a numeric encoding was developed to provide a minimized representation of genome sequences, which was used to approximate genetic distances. Based on approximate genetic distances, the most relevant sequence pairs were identified and used to create sparse genetic distance matrices. In this context, it was also observed how different candidate search strategies influence the visualizations of outbreak dynamics. For very large numbers of sequences, \acrlong{anns} was used to find relevant sequence pairs while overcoming quadratic scaling to generate visualizations more efficiently.

Chapter \ref{cha:background} provides a comprehensive overview of the background related to the applied methodology and implementations. This includes how disease outbreaks are analyzed using traditional epidemiological methods, the integration of genetic information into this process, and the visual representation of disease outbreak dynamics using graphs. Subsequently, it describes how GENTRAIN implements outbreak analysis with a focus on the underlying algorithmic procedures. Furthermore, the relevance of genomics in epidemiology, as well as efficient methods for calculating genetic distance matrices and visualizing outbreak dynamics on a large scale, are examined in existing research.
The methodology applied is described in Chapter \ref{cha:methodology}. This includes a description of the dataset utilized in this work, the identification of relevant data aggregates to represent plausible outbreak scenarios, the encoding process of genome sequences, the modification as well as the approximation of the extensive GENTRAIN distance calculation, and the identification of the most relevant candidate pairs. It is also outlined how the run time of the applied approaches and the preservation of the interpretability for outbreak analysis are evaluated.
The results of this work are presented in Chapter \ref{cha:results} and interpreted in Chapter \ref{cha:discussion}, taking into account their limitations and applicability. Chapter \ref{cha:conclusion} summarizes this work and provides suggestions on how the results can be used for further scientific research in this field.