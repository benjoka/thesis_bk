\section{Conclusion}
\label{cha:conclusion}
This work addressed the minimization of computational effort involved in visualizing the dynamics of large-scale viral outbreaks while preserving the interpretability for outbreak analysis. Therefore, the applicability of genomic investigations for analyzing viral outbreaks was emphasized, and the importance of making this procedure accessible to non-specialists through user-oriented applications was demonstrated based on related literature and implemented solutions. 

The primary factors contributing to the extensive runtime during the calculation of the genetic distances are the quadratic scaling when calculating complete genetic distance matrices and the generation of \acrshortpl{mst} for dense matrices. The quadratic scaling was addressed through two distinct approaches. First, the execution time for individual genetic distance calculations was reduced through the modification of the GENTRAIN algorithm and the approximation of these results. Secondly, the necessity of calculating genetic distances between all possible sequence pairs was circumvented through \acrshort{ccs} and \acrshort{acs}. Although the first measure aims to mitigate the effects of the quadratic scaling, the second measure truly prevents it.
The increase in runtime when generating \acrshortpl{mst} for dense genetic distance matrices was also addressed by selective candidate calculation. 

The evaluation of the deviation of the genetic distance matrices resulting from the modified algorithm and the genetic distance approximation showed strong rank-based correlations and low error, especially for infectious distances. Furthermore, the infection recall indicated a high accuracy in preserving infections within distance matrices. Although the clusters showed high purity in terms of genomic lineage and a strong relationship with the dates and cities of sample collection, the sequential connections between the separate clusters were not completely preserved. An explanation for this could be the relatively low distance edges connecting these clusters, which might yield high sensitivity in terms of the constellation of cluster chains. In general, the quality of the results regarding the preservation of infection chains and the representation of outbreak contexts increased with the seasonal period and geographic area. This was explained by lower infection rates, which produce a higher deviation of the genetic distance and make the genetic distances more distinguishable. Since higher dimensions in terms of seasonal period and geographic area yield more sequences, these scenarios might also be of greater relevance for large-scale outbreak analyses. In summary, it could be concluded that preserving the interpretability potential of viral outbreak analysis while keeping the computational effort manageable was a realistic objective. 

When comparing the approaches proposed by this work, very different runtime savings and results were achieved. The modified algorithm yielded significant runtime savings of about 85~\% while identifying around 98~\% of present infections. The resulting \acrshortpl{mst} also showed high pure clusters in terms of the lineage association and comparable edge weight distributions compared to $T$\textsubscript{gen}. \acrshort{ccs} led to even higher runtime savings, which depended heavily on the calculation rates used. Compared to the modified algorithm, \acrshort{ccs} with a calculation rate of 0.2 was capable of saving additional 76~\% of runtime for aggregates comprising 5,000 sequences, while identifying 93~\% of infections and preserving the \acrshort{mst} structure to a comparable  extent as the modified algorithm. Finally, \acrshort{acs} demonstrated substantial runtime savings of approximately 95~\% compared to the modified algorithm in a scenario that was particularly vulnerable to quadratic scaling due to a large sample size of 10,000. Furthermore, \acrshort{acs} required only half the runtime of \acrshort{ccs} while achieving comparable results.

Concerning larger datasets, an implementation of the proposed algorithmic adjustments could be valuable for GENTRAIN. These adjustments have been shown to significantly decrease runtime while preserving a high degree of interpretability for outbreak analysis. Furthermore, \acrshort{ccs} or \acrshort{acs} could be used conditionally for even larger-scale outbreak analysis scenarios. In order for other viruses to be applicable, the corresponding reference genomes must be collected. GENTRAIN already provides these references for several supported pathogens, including Morbillivirus hominis and Varicella zoster virus.
However, as the mutation encoding is adjusted towards the mutation behaviors of SARS-CoV-2 and the GENTRAIN algorithm, further refinement may be necessary for other viruses or algorithms. Subsequent research is also needed on acceptance benchmarks for runtimes in user-oriented applications within the context of genetic outbreak analysis. 
As outlined in Section \ref{cha:related_work}, \acrshort{anns} is widely discussed in the fields of large-scale genome comparison and has already shown promising results, of which this work could only provide a glimpse. Further research on appropriate \acrshort{anns} methods to preserve interpretability for outbreak analysis could complement the present work.

It remains uncertain whether real-time viral outbreak analyses will ever reach the point where large-scale outbreak scenarios, as discussed in this work, become reality. Although the systematic relevance of such analyses is not desirable, failing to incorporate scientific contributions when preparing for these scenarios would be even less desirable.
