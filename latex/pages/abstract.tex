\section*{Abstract}
The growing significance of disease outbreak analysis, in conjunction with enhanced genome sequence availability, necessitates the development of user-oriented applications that facilitate genetic outbreak analysis for non-specialists. Since the computational effort involved in visualizing genetic outbreak dynamics increases significantly with the number of sequences considered, this work aims to reduce the effort required to visualize outbreak dynamics while preserving interpretability for outbreak analysis. Key contributions include the algorithmic modification of single genetic distance calculations, as well as the calculation of sparse genetic distance matrices through genetic distance approximation and approximate nearest neighbor search. The proposed approaches exhibit a high degree of accuracy in preserving infection chains and outbreak-related contexts, while significantly reducing the necessary runtime. In general, longer seasonal periods and wider geographic areas tend to yield superior preservation of outbreak characteristics.

\section*{Zusammenfassung}
Die zunehmende Bedeutung der Analyse von Infektionsketten sowie die verbesserte Verfügbarkeit von Genomsequenzen unterstreichen die Notwendigkeit nutzerfreundlicher Anwendungen, die genetisch gestützte Ausbruchsanalysen vereinfachen. Aufgrund der signifikanten Erhöhung des notwendigen rechnerischen Aufwands zur Visualisierung genetischer Ausbruchsszenarien ist das Ziel dieser Arbeit, diesen Aufwand zu reduzieren und gleichzeitig die Interpretierbarkeit für Ausbruchsanalysen bestmöglich beizubehalten. Zu diesem Zweck wurde die Berechnung der genetischen Distanz optimiert und selektive Berechnungen genetischer Distanzmatrizen auf Basis von Distanzapproximationen sowie mittels Approximate Nearest Neighbor Search durchgeführt.
Die vorgestellten Methoden bewahren Infektionsketten und ausbruchsbezogene Kontexte weitgehend, während die notwendige Laufzeit drastisch reduziert wird. Grundsätzlich ist festzuhalten, dass die Betrachtung längerer Zeiträume und größerer geografischer Bereiche die Erhaltung von Ausbruchs-Charakteristika fördert.